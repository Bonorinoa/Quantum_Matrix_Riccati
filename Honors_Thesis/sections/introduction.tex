\chapter{Introduction} Count Jacopo Francesco Riccati (1676-1754) is the mathematician, along with Giovanni Poleni and Bernardino Zendrini, credited with developing the now fundamental family of Riccati Differential Equations \cite{jungers_historical_2019}. Interested in problems related to plane curves determined by curvature properties, Riccati showed that the solution curve arises from a second order differential equation after applying a suitable change of variables \cite{riccati_soluzione_nodate}. This discovery led him to four particular solutions, which arguably marked the beginning of the famous Riccati equations. Eager to find more solutions to his equation, Riccati decides to reach out to his mentor, Poleni, asking to share his work with Nicolas I Bernoulli \cite{riccati_letter_1718} and Christian Goldbach. Goldbach’s and Bernoulli's family expertise played a crucial role in further developing methods to solve this new interesting equation that Francesco Riccati has proposed. Work that culminated in a paper published in 1724 titled “Animadversiones in aequationes differentiales secundi gradus”, by Jacopo F. Riccati \cite{riccati_animadversiones_1724}.
